%!TEX TS-program = pdflatex
\documentclass[12pt, b6paper, openany]{memoir}

    \usepackage[cm]{fullpage}
    \setstocksize{182mm}{128mm}
    \usepackage{enumitem}
    \usepackage[paperwidth=128mm, paperheight=182mm, top=1.5cm, bottom=2.2cm, inner=2cm, outer=2.5cm]{geometry}
    \usepackage{titlesec}
    \usepackage{kotex}
    \usepackage[breaklinks=true]{hyperref}
    \hypersetup{colorlinks, citecolor=blank, filecolor=blank, linkcolor=blank, urlcolor=blank}
    
    \usepackage{epigraph}
    \setlength\epigraphwidth{1\textwidth}
    
    \makeatletter
    \newcommand*{\cleartoleftpage}{\clearpage\if@twoside\ifodd\c@page\hbox{}\newpage\if@twocolumn\hbox{}\newpage\fi\fi\fi}
    \makeatother
    
    \makeatletter
    \newcommand*{\cleartorightpage}{\clearpage\if@twoside\ifeven\c@page\hbox{}\newpage\if@twocolumn\hbox{}\newpage\fi\fi\fi}
    \makeatother
    
    \usepackage{titling}
    %define titlingpage
    \pretitle{\begin{flushleft}\begin{normalsize}\begin{textbf}}
    \posttitle{,}
    \preauthor{\end{textbf}}
    \postauthor{,}
    \predate{}
    \postdate{\end{normalsize}\end{flushleft}}
    % Set title of tableofcontent
    \renewcommand{\contentsname}{차례}
    % set tableofcontent
    \maxtocdepth{chapter}
    \renewcommand{\baselinestretch}{1.25}
    \setsecnumdepth{part}
    \setlength{\beforechapskip}{0pt}
    
    \titleformat{\chapter}{\filright}{}{0pt}{\normalfont\large\bfseries}
    \titlespacing*{\chapter}{0pt}{0pt}{2\baselineskip}
    
    \setlength{\parskip}{1em}
    
    \newenvironment{lastnote}{%
        \clearpage\vspace*{\fill}%
        \begin{footnotesize}
    }{%
        \end{footnotesize}
    }
    
    \newenvironment{lyric}{\setlength{\parindent}{0pt}}{}
    \newenvironment{article}{}{}
    
    
    
    
    \setcounter{secnumdepth}{0}
    
    % Footnotes: 
    
    
    % Title, authors, date.
    \title{이상기후 샘플북}
    \author{에이미 앰플}
    \date{2018}
    
    \begin{document}
    
    \frontmatter
    \begin{titlingpage}
    \maketitle
    \end{titlingpage}
    
    \cleartorightpage
    
      \tableofcontents
      
    \mainmatter
    \begin{lyric}
    \hypertarget{uxc624uxb298uxc758-uxb0a0uxc528}{%
    \chapter{오늘의 날씨}\label{uxc624uxb298uxc758-uxb0a0uxc528}}
    
    그날 해가 뜨지 않았어\\
    어젠 구름도 끼지 않았어\\
    바람조차 불지 않고 비는 내리지 않았어\\
    오늘은 춥지 않았어\\
    덥지 않았어\\
    눈도 내리지 않았어\\
    내일은 날씨가 어떠니?
    
    아마도 내일 난 택시를 타고 러시아에 갈 거야\\
    아마도 영원히 지지 않은 태양 가운데 있을 거야 아마도 끝나지 않을 눈 위에 묻힐 거야\\
    아마도 바람의 끝에 엎드려 있을 거야\\
    아마도 난\\
    따뜻한 헤드폰을 붙잡고 있을 거야\\
    그동안 우릴 피해 갔던 모든 것들을 놓칠 거야
    
    택시 안에서 붉은 석양이 보이면 내일 날씨를 들려줄게
    \end{lyric}
    
    \begin{lyric}
    \hypertarget{uxd0dcuxc591uxc758-uxb300uxc9c0}{%
    \chapter{태양의 대지}\label{uxd0dcuxc591uxc758-uxb300uxc9c0}}
    
    살찐 태양이 대지에 눌러앉았던 어느 여름이었다 발치에 치이는 계절을 걷어차며 무릎까지 쌓인 계절에 묻힌 발을 꺼내는 것이 매일의 일이었다. 옆집의 미친 남자는 이때다 싶어 20대 여성을 죽여 그 아래 묻어두었다. 나의 친구는 내게 안부를 묻곤 한다. 우린 언제쯤 만날 수 있을까. 반도를 타고 낮게 흐르는 이 빛은 사라질까. 어떤 이들은 그가 가장 사랑하는 것을 박살 내고 어떤 이들은 그가 가장 미워하는 것을 박살 낸다. 그가 지난해 가장 사랑했던 것은 엑스박스였다.
    
    모두가 이 나라를 떠나고 싶어 했지만 계절은 한 세기 동안 정체되었고 위대한 상상은 해안가에 머물렀다 우리가 상상할 수 있는 건 이곳이 아닌 여기까지였다
    
    그곳은 어떤 계절이니? 여긴 열 세 번째 계절이야. 그곳은 어떤 계절이니? 이곳은 낮은 두 번째 계절이지. 한 세기 전을 기억하는 유일한 인류가 이번 세기를 버티고 있다 빌딩 옥상까지 닿은 이 땅의 계절은 유리창을 두드리며 죽은 여자의 음성을 전했다 이 계절의 쓸모가 드러나는 유일한 순간이었다 유리창이 박살나 정수리에 피를 흘리는 여자를 두고 땀을 질질 흘리던 우리는 너 때문이라 탓했다 이 행성의 껍데기가 떨리며 우리 음성을 전 세계에 전했다 우주는 우리와 한편이었어 적어도 이 행성만큼은 그 여자는 죽고 나서야 제 편을 가질 수 있었다
    
    여자들은 모두 계절의 가장 낮은 곳에서 마주쳐 우리가 되었고, 계절에 밀려 행성 밖으로 밀려났다 우리가 넘어서지 못한 해안선의 바깥으로
    
    높아진 우리는 늘어진 시간 위에서 더 많은 불면의 새벽을 보낼 것이다 창가의 여자들은 모조리 대지 위에 서겠지 결국 끝까지 살아남는 건, 이번 해를 살아서 버티는 건 나는 결코 아닐 것이다 너 또한 아닐 것이다
    
    이번 해도 겨울은 이 행성을 지나쳐 가버렸다 높은 내 정수리엔 찰나의 서리가 녹아 흐른다 모두들 행운을 빈다
    \end{lyric}
    
    \begin{lyric}
    \hypertarget{uxc5ecuxb984uxc758-uxc774uxc57cuxae30}{%
    \chapter{여름의 이야기}\label{uxc5ecuxb984uxc758-uxc774uxc57cuxae30}}
    
    뜨거운 여름이었다\\
    여름의 정상에 다다른 때였을 것이다\\
    확실친 않지만 그는\\
    그늘을 찾아 돌아다녔던 것 같다 다행히\\
    데오도란트를 뿌렸지만\\
    백팩을 멘 등이 까맣게 젖을까 봐\\
    한쪽 어깨로만 백팩을 메고 있었다
    
    그는 우린\\
    항상 이렇게\\
    모든 이야기를 시작하곤 했다\\
    오늘 날씨는 뜨겁고 화창하다고
    
    목소리를 다시 내려보낼 구름 같은 건 없었고\\
    전리층은 은하 바깥으로 퍼져버렸다\\
    한 세기가 지나기 전엔\\
    응답을 들을 수 있겠지 아마도\\
    이 모든 구호는 내일의 구름이 될 거라고\\
    뭐 그런 거라고
    
    한 세기가 지나기 전엔\\
    이런 아침의 다짐.
    
    그는 대답했다\\
    나는 잘 지내\\
    결국 그는 그렇게 대답을 하게 된다
    
    나는 잘 지내
    
    우린 항상 모든 이야기를 이렇게 끝내곤 했다
    
    나는 잘 지내
    
    다음 해가 되기 전에\\
    대답 할 수 있길 바라며\\
    나는 잘 지내
    \end{lyric}
    
    \backmatter
    
    \begin{lastnote}
    \begin{description}[itemsep=1pt,parsep=1pt]%
    \item[제목]%
    이상기후 샘플북%
    \item[저자]%
    에이미 앰플
    \item[편집]%
    미루
    \item[디자인]%
    써드엔지니어링카르텔
    \item[출간일]%
    2018-08-12%
    \end{description}
    
    \begin{description}[itemsep=1pt,parsep=1pt]%
    \item[출판]%
    금치산자레시피
    \item[이메일]%
    gtsz.rcp@gmail.com
    \item[웹사이트]%
    http://gtszrcp.com
    \item[인스타그램]%
    gtsz.rcp
    \end{description}
    
    \begin{description}[itemsep=1pt,parsep=1pt]%
    \item[저작권]%
    이 책에 수록된 저작물 중 별도로 표기되지 않은 모든 저작물의 저작권은 저자에게 있습니다. 크리에이티브커먼즈 저작자표시-동일조건변경허락 4.0 국제 라이센스
    \end{description}
    \end{lastnote}
    \end{document}