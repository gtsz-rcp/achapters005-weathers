\begin{article}
\hypertarget{uxc624uxb298uxc758-uxb0a0uxc528}{%
\chapter{오늘의 날씨}\label{uxc624uxb298uxc758-uxb0a0uxc528}}

그날 해가 뜨지 않았어\\
어젠 구름도 끼지 않았어\\
바람조차 불지 않고 비는 내리지 않았어\\
오늘은 춥지 않았어\\
덥지 않았어\\
눈도 내리지 않았어\\
내일은 날씨가 어떠니?

아마도 내일 난 택시를 타고 러시아에 갈 거야\\
아마도 영원히 지지 않은 태양 가운데 있을 거야 아마도 끝나지 않을 눈 위에 묻힐 거야\\
아마도 바람의 끝에 엎드려 있을 거야\\
아마도 난\\
따뜻한 헤드폰을 붙잡고 있을 거야\\
그동안 우릴 피해 갔던 모든 것들을 놓칠 거야

택시 안에서 붉은 석양이 보이면 내일 날씨를 들려줄게
\end{article}

\begin{article}
\hypertarget{uxc608uxc815uxbcf4uxb2e4-uxbe60uxb978-uxc2dcuxac01}{%
\chapter{예정보다 빠른 시각}\label{uxc608uxc815uxbcf4uxb2e4-uxbe60uxb978-uxc2dcuxac01}}

예정보다 빠른 시각\\
너와 같은 계절이 지나쳤다\\
그때도 난 안전한 자리에 섰다\\
구겨진 자리가 숨을 들이쉰다\\
얼마 지나지 않아\\
다음 지하철이 도착할 것이다\\
한 임산부가 안전한 경계에 선다\\
구겨진 자리가 숨을 내쉰다

예정보다 빠른 시각\\
의심스러운 출산이 시작되려 한다\\
겨울 같은 계절이다\\
저마다 여름 나무가 싹 트는 봄을 이야기한다\\
겨울과 다른 계절이다\\
우리가 할 수 있는 말은\\
이미 지는 여름이었다\\
제한된 화장실이 덜컹거린다\\
덜 마른 타일이 떨어져 박살 난다
\end{article}

\begin{article}
\hypertarget{uxd558uxb4dcuxb79cuxb529}{%
\chapter{하드랜딩}\label{uxd558uxb4dcuxb79cuxb529}}

우린 가방을 짊어지고 재빨리 걸어갔다\\
어쨌든 오늘 구름은 해를 가리지 않았다\\
어쩌면 내일 우리는 버스 정류장 앞에서 점심을 먹을 것이다

신호등을 기다리며 우린 재빨리 키스했다 그리고 맥심 커피를 타 마셨고 재빨리 설탕을 씹어 삼켰다\\
아스팔트가 발등에 눌어붙는다

바보가 아닌 이상 우린 이 풍경을 경험한 적도 없을 것이다\\
우린 단 한번도 풍경을 이야기한 적 없다\\
우린 단 한번도 이야기한 적이 없었다

가방을 내려놓지 않는 이상 우린 어떤 이야기도 할 수 없으니까\\
가방을 짊어지고 재빨리 걸어갔다

어쩌면 우린 내일 잔디 광장 가운데 착륙한 지하철에\\
아스팔트로 눌어붙은 낙서를 하고 있을 것이다
\end{article}

\begin{article}
\hypertarget{uxd0dcuxc591uxc758-uxb300uxc9c0}{%
\chapter{태양의 대지}\label{uxd0dcuxc591uxc758-uxb300uxc9c0}}

살찐 태양이 대지에 눌러앉았던 어느 여름이었다 발치에 치이는 계절을 걷어차며 무릎까지 쌓인 계절에 묻힌 발을 꺼내는 것이 매일의 일이었다. 옆집의 미친 남자는 이때다 싶어 20대 여성을 죽여 그 아래 묻어두었다. 나의 친구는 내게 안부를 묻곤 한다. 우린 언제쯤 만날 수 있을까. 반도를 타고 낮게 흐르는 이 빛은 사라질까. 어떤 이들은 그가 가장 사랑하는 것을 박살 내고 어떤 이들은 그가 가장 미워하는 것을 박살 낸다. 그가 지난해 가장 사랑했던 것은 엑스박스였다.

모두가 이 나라를 떠나고 싶어 했지만 계절은 한 세기 동안 정체되었고 위대한 상상은 해안가에 머물렀다 우리가 상상할 수 있는 건 이곳이 아닌 여기까지였다

그곳은 어떤 계절이니? 여긴 열 세 번째 계절이야. 그곳은 어떤 계절이니? 이곳은 낮은 두 번째 계절이지. 한 세기 전을 기억하는 유일한 인류가 이번 세기를 버티고 있다 빌딩 옥상까지 닿은 이 땅의 계절은 유리창을 두드리며 죽은 여자의 음성을 전했다 이 계절의 쓸모가 드러나는 유일한 순간이었다 유리창이 박살나 정수리에 피를 흘리는 여자를 두고 땀을 질질 흘리던 우리는 너 때문이라 탓했다 이 행성의 껍데기가 떨리며 우리 음성을 전 세계에 전했다 우주는 우리와 한편이었어 적어도 이 행성만큼은 그 여자는 죽고 나서야 제 편을 가질 수 있었다

여자들은 모두 계절의 가장 낮은 곳에서 마주쳐 우리가 되었고, 계절에 밀려 행성 밖으로 밀려났다 우리가 넘어서지 못한 해안선의 바깥으로

높아진 우리는 늘어진 시간 위에서 더 많은 불면의 새벽을 보낼 것이다 창가의 여자들은 모조리 대지 위에 서겠지 결국 끝까지 살아남는 건, 이번 해를 살아서 버티는 건 나는 결코 아닐 것이다 너 또한 아닐 것이다

이번 해도 겨울은 이 행성을 지나쳐 가버렸다 높은 내 정수리엔 찰나의 서리가 녹아 흐른다 모두들 행운을 빈다
\end{article}

\begin{article}
\hypertarget{uxc5ecuxb984uxc758-uxc774uxc57cuxae30}{%
\chapter{여름의 이야기}\label{uxc5ecuxb984uxc758-uxc774uxc57cuxae30}}

뜨거운 여름이었다\\
여름의 정상에 다다른 때였을 것이다\\
확실친 않지만 그는\\
그늘을 찾아 돌아다녔던 것 같다 다행히\\
데오도란트를 뿌렸지만\\
백팩을 멘 등이 까맣게 젖을까 봐\\
한쪽 어깨로만 백팩을 메고 있었다

그는 우린\\
항상 이렇게\\
모든 이야기를 시작하곤 했다\\
오늘 날씨는 뜨겁고 화창하다고

목소리를 다시 내려보낼 구름 같은 건 없었고\\
전리층은 은하 바깥으로 퍼져버렸다\\
한 세기가 지나기 전엔\\
응답을 들을 수 있겠지 아마도\\
이 모든 구호는 내일의 구름이 될 거라고\\
뭐 그런 거라고

한 세기가 지나기 전엔\\
이런 아침의 다짐.

그는 대답했다\\
나는 잘 지내\\
결국 그는 그렇게 대답을 하게 된다

나는 잘 지내

우린 항상 모든 이야기를 이렇게 끝내곤 했다

나는 잘 지내

다음 해가 되기 전에\\
대답 할 수 있길 바라며\\
나는 잘 지내
\end{article}

\begin{article}
\hypertarget{uxbd04}{%
\chapter{봄}\label{uxbd04}}

순식간에 가을이 스쳐 지나가고, 미처 추운 겨울을 준비하지 못한 나무들은 모두 얼어 죽었다.

그리고 그 해.\\
서울에 눈이 삼십 미터나 왔다. 눈은 도시 허공까지 몽땅 채워 사람들은 아래로 내려와 긴 휴가를 보냈다. 우린 몽당연필처럼 삼십 미터 눈 위에 나란히 꽂혀 있었다.\\
하늘이 좀 더 가까운 이곳은 여전히 더 추웠고, 우리 사이의 공기도 겨울의 영하를 피해갈 수 없었다.\\
난 봄을 발음했다. 그녀는 하늘보다 파랬던 두 손목으로 날 안아주었다 두 손목은 조금 따뜻했다.\\
난 그녀 품 안에서 눈이 빨리 녹아내리길 기대했다.

눈이 녹아 우리가 아스팔트 한가운데 있을 때\\
그녀가 미처 멈추지 못한 차에 치여 숨졌을 때.\\
봄이 왔다.\\
긴 휴가가 끝난 사람들은 다시 허공으로 올라갔다.

아무도 없는 바닥에 작년에 보았던 짧은 들풀을 발견했다.\\
봄이 왔다.
\end{article}

\begin{article}
\hypertarget{uxb208-uxb179uxc73cuxba74}{%
\chapter{눈 녹으면}\label{uxb208-uxb179uxc73cuxba74}}

몇 달이 지나고 결국 눈은 녹아내리겠지\\
진실이 묻힌 자리 우린 나란히 서 있겠지\\
드러난 진실은 결국 내게 책임을 묻겠지\\
몇 달의 시간은 오늘의 눈처럼 높이 쌓였지\\
우린 맘 놓고 시간에 파묻혀\\
진실을 외면 하리라 마음먹지 않아도\\
어차피 진실은 아래 묻혀 있으니\\
낮아진 태양과 좀 더 가까워지길 기다렸었지 우린 언제고 이 눈이 내리길 기도했어\\
기도는 항상 어처구니없는 바람을 낳고\\
애꿎은 손목만 꺾여 더는 펴지지 않았어\\
세계의 추위 속에서\\
우린 세계의 동맥이 된 양\\
파랗게 박동 하길 멈추지 않았지\\
눈이 녹고:\\
우린 진실과 마주하겠지\\
나는 진실이 되어 있겠지\\
너는 진실을 알게 되겠지\\
우린 진실과 마주하겠지
\end{article}

\begin{article}
\hypertarget{uxbbf8uxb4e4uxc0ccuxd504uxb780uxc2dcuxc2a4uxcf54}{%
\chapter{미들샌프란시스코}\label{uxbbf8uxb4e4uxc0ccuxd504uxb780uxc2dcuxc2a4uxcf54}}

이번 겨울에 난 미들 샌프란시스코로 갈거야. 겨울이 시작되기 불과 몇 시간 전이었다. 곧 달이 뜨면 차가운 겨울이 바다에서 기어 나올 거야. 한국은, 서울은 너무 추워. 보일러를 가동해야 할 집주인은 항상 머리 꼭대기까지 이 가득 차고 나서야 보일러를 틀었다. 기름값이 올라서 비행기가 뜨지 못할 수도 있는데, 상관없어. 그곳에 가는 동안 난 아주 작아질 테니까. 사과만큼. 그곳엔 이 오지 않는데. 하지만 한국처럼 추울지도 몰라. 난 반팔 옷을 입고 떨면서 겨울 같은 노래를 부르겠지. 난 거기서 아메리카노 팩을 빨고 있을 거야. 이번 겨울은 너무 추워. 작년엔 눈이 오지 않았어.

난 그의 말을 가로막고 달력을 펴 하루하루를 읽어주었다. 침대 위에 앉은 내 손등 근처로 보리차를 쏟았다. 뜨거운 보리차가 침대를 적신다. 침대에선 희미한 땀 냄새와 비누 냄새와 샴푸 냄새가 났고, 셰이빙 폼 냄새가 났다. 그때가 1년 전이었다.

그의 문을 두드렸을 때. 난 이미 그의 방 열쇠를 쥐고 있었다. 그날 겨울은 달이 뜨자마자 보리차가 식듯 하루도 안 되어 사라졌다. 난 트렁크에 들어가 미들 샌프란시스코로 갈 거야. 이번 겨울에 눈이 내리지 않은 건 이 방이 3층에 있기 때문일 거야. 그는 슬퍼했다. 겨울은 높은 곳까지 올라오지 못했다. 난 내년 겨울 공항에서 천천히 작아질 것이다. 그는 이렇게 노트에 써 놓았다.

오늘도 난 그를 위해 내일을 불러 주려 했다. 인기척 없는 두꺼운 문을 열고 들어간 방에는 뜨거운 보리차로 가득 찬 욕조가 놓여 있었다. 그 곁에 전기 포트가 있었고, 텅 빈 생수통이 있었다. 그리고 천장에 밧줄이 메여 있었다.

부츠와 스타킹을 벗고 티백이 가득 찬 욕조에 들어갔다. 난 욕조에 걸터앉아 밧줄을 살펴보았다. 아마도 그는 지문조차 남기지 않고 사라졌을 것이다. 아마 이번 겨울엔 눈이 내릴 것이다. 아마 미들 샌프란시스코는 따뜻할 것이다. 아메리카노 팩은 빨대를 꽂아 마시는 걸까? 트렁크는 따뜻하겠지? 그리고 다음 날.

그에게 존재하지 않는 다음날을 오늘 읽어주려 했는데, 하루만 더 서울에서 살아있길 바랐는데. 난 그를 조용히 내려놓았다. 빨간 트렁크 안에는 팬티 두 개와 브래지어 두 개, 스타킹 두 켤레와 티셔츠 다섯 벌, 코트 하나, 생리대와 여행용 선물세트 하나, 세 가지 색의 립스틱이 구두 여섯 켤레 옆에 있었고, 한 켠에 달력이 있었다. 그 틈에 그를 뉘었다.

이제 곧 달이 뜨고 먼 바다에서 겨울이 오는 소리가 들릴 것이다.
\end{article}
